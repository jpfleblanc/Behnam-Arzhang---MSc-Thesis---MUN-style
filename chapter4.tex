\chapter{Conclusion}

In this thesis, we studied the 2D Hubbard model using many-body theory. We used DMFT and DF methods to solve and describe the system with high momentum resolution. Calculations based on cluster extensions of dynamical mean-field theory (DCA, cDMFT), typically assume the system to be paramagnetic and suppress anti-ferromagnetic correlations during the self consistency. This is because of this assumption that in the limit of infinite cluster size the system wont have a second-order phase transition to an anti-ferromagnetic state. What was not expected is that anti-ferromagnetic spin-correlations with a finite correlation length scale could be sufficient to lead to a nFL state. It appears that small-cluster cDMFT or DCA calculations in the weak coupling regime and at low-temperatures would lead to an incorrect representation of the correlation length-scale due to a truncation of the system size. The question of precisely how these spin fluctuations cause suppression of states near the
Fermi level has not been answered. To discuss this problem, we used a recent development method called fluctuation diagnostics. We used the two-particle vertex function to decompose single-particle self-energy into various basis representations and scattering channels. We represented our results in the spin channel. 

In chapter 1, we presented an overview of strongly correlated phenomena, the Hubbard model and Fermi liquid theory. In chapter 2, the concept of many-body theory and numerical techniques which we used throughout the work has been provided. Dynamical Mean Field Theory $(DMFT)$, the Continuous-Time Auxiliary Field algorithm $(CT-AUX)$, Dual Fermions method $(DF)$ which introduce non-local correlations and provides a cheap method to obtain high resolution momentum space information, Susceptibility and vertex function $F$ by using inverse Bethe-Salpeter Equation has been found, Fluctuation Diagnostics and the Maximum Entropy Method were all explained in this chapter.

In chapter 3, we presented our results for studying the 2D Hubbard Model on a square lattice. We described how to perform DMFT simulations on this system and use the output of DMFT as input for DF codes. We presented the result of density obtained by DMFT and Dual fermion method. Then we discussed the spectral functions which we found for both cases of DMFT and Dual Fermions results. We presented the self-energy for DMFT output as well as Nodal and Anti-nodal points obtained from Dual fermion method. We could see the transition from metal to an insulator by studying self-energy. We showed the relation between $\Delta\Sigma$ and density for different energy in Nodal and Anti-nodal region. The correlation length has been found for different energies. By Fluctuation Diagnostics method we access to any frequency we need so we showed $\Sigma$ for two first frequencies. We also showed the $\Delta \Sigma$ dependence to $(q_x, q_y)$.

DMFT and DF helped us to find Green's function, Self-energy, and spin susceptibility function of  Hubbard model on 2D square lattice. While DMFT results are momentum independent DF provides us with momentum dependent results with reasonable computational cost and high resolution momentum objects. In fact, we are able to produce high-resolution quantity in momentum space $(64 \times64 Kp)$ and observe the structure of first BZ in more details, while other researchers by using DCA method only could use 8-site calculation \cite{Toschi}. Our results $(64\times 64 Kp= 4096)$ in comparison with DCA research on $8-sites$, our system is $512$ times larger, while it is faster and not as expensive as DCA method. The spectral function obtained by DF shows phase transition in smaller $U$ in comparison with DMFT results. Moreover, we showed that zeroth Bosonic Frequency has the most contribution in the Self-energy function.